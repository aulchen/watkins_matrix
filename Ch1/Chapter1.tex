\documentclass{article}
\usepackage{amsmath}
\usepackage{amsfonts}
\usepackage{amssymb}
\usepackage{algpseudocode}

\newenvironment{proof}{\paragraph{Proof:}}{\hfill$\square$}
\newtheorem{theorem}{Theorem}
\newtheorem{lemma}[theorem]{Lemma}
\newtheorem{corollary}[theorem]{Corollary}

\author{Arthur Chen}
\title{Watkins Matrix Computations Chapter 1}
\date{\today}

\begin{document}

\section*{1.3 Triangular Systems}

\subsection*{Exercise 1.3.15}

Develop row-oriented back substitution for upper-triangular matrices.

Let the system be $Ux = y$, where $U$ is upper triangular. Writing out the equations,

\begin{align*}
u_{11}x_1 + u_{12}x_2 \dots u_{1,n-1}x_{n-1} + u_{1n}x_n &= y_1 \\ 
u_{22}x_2 \dots u_{2,n-1}x_{n-1} + u_{2n}x_n &= y_2 \\
\vdots \\
u_{n-1,n-1}x_{n-1} + u_{n-1,n}x_n &= y_{n-1} \\
u_{nn}x_n &= y_n
\end{align*}

This suggests the following algorithm.
TODO.

\section*{1.7 Gaussian Elimination and the LU Decomposition}

\subsection*{Exercise 1.7.2}

Prove Proposition 1.7.1: If $\hat{A}x=\hat{b}$ is obtained from $Ax = b$ by an elementary operation of type 1, 2, or 3, then the systems $Ax = b$ and $\hat{A}x=\hat{b}$ are equivalent. That is, they have the same solution set.

\begin{proof}
We begin with showing operations of type 1, i.e., adding a multiple of one equation to another equation. Let $A_i$ be the $i$th row of matrix $A$.

For the forward, we need to show that if $x$ solves $Ax = b$, then $x$ solves $\hat{A}x = \hat{b}$. Suppose that $\hat{A}$ was created by adding $m$ times row $p$ to row $q$ of $A$. Then $\hat{A}_i = A_i$ and $\hat{b}_i = b_i$ for $i \neq q$, and $\hat{A}_q = mA_p + A_q$, $\hat{b}_q = mb_p + b_q$. $x$ solves $\hat{A}x = \hat{b}$ for rows $i \neq q$ trivially. For row $q$, $\hat{A}_q x = (mA_p + A_q) x = mA_p x + A_q x = mb_p + b_q = \hat{b}_q$. Thus $x$ solves $\hat{A}x = \hat{b}$.

For the reverse, we need to show that if $x$ solves $\hat{A}x = \hat{b}$, then $x$ solves $Ax = b$. Again, suppose that $\hat{A}$ was created by adding $m$ times row $p$ to row $q$ of $A$. Since $\hat{A}_p = A_p$, by subtracting $m$ times row $p$ from row $q$ of $\hat{A}$, we get back $A_q$ and $b_q$. Since this is an elementary row operation of type 1, the theorem proved above holds.

Interchanging two rows and multiplying an equation by a non-zero constant are trivial.
\end{proof}

\end{document}